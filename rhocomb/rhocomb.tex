\documentclass{llncs}
\usepackage{amsmath}
\usepackage{amssymb}
\usepackage{comment}
\usepackage{hyperref}
\usepackage{longtable}
\usepackage{stmaryrd}
\newcommand{\interp}[1]{\llbracket #1 \rrbracket}
\newcommand{\maps}{\colon}
\newcommand{\FinSet}{\mathrm{FinSet}}
\newcommand{\Set}{\mathrm{Set}}
\newcommand{\Cat}{\mathrm{Cat}}
\newcommand{\Calc}{\mathrm{Calc}}
\newcommand{\Mon}{\mathrm{Mon}}
\newcommand{\BoolAlg}{\mathrm{BoolAlg}}
\renewcommand{\Form}{\mathrm{Form}}
\newcommand{\leftu}{\mathrm{left}}
\newcommand{\rightu}{\mathrm{right}}
\newcommand{\send}{\mathrm{send}}
\newcommand{\recv}{\mathrm{recv}}
\newcommand{\comm}{\mathrm{comm}}
\renewcommand{\quote}[1]{``#1"}
\newcommand{\deref}[1]{\mathrm{eval}(#1)}
\newcommand{\op}{\mathrm{op}}
\newcommand{\NN}{\mathbb{N}}
\makeatletter
\gdef\tshortstack{\@ifnextchar[\@tshortstack{\@tshortstack[c]}}
\gdef\@tshortstack[#1]{%
  \leavevmode
  \vtop\bgroup
    \baselineskip-\p@\lineskip 3\p@
    \let\mb@l\hss\let\mb@r\hss
    \expandafter\let\csname mb@#1\endcsname\relax
    \let\\\@stackcr
    \@ishortstack}
\makeatother

\title{Logic as a distributive law}
\author{
Michael Stay\inst{1}\\
\and
L.G. Meredith\inst{2}\\
}
\institute{
  {Pyrofex Corp.}\\
  \email{\fontsize{8}{8}\selectfont stay@pyrofex.net}\\
  \and
  {RChain Cooperative}\\
  \email{\fontsize{8}{8}\selectfont greg@rchain.coop}
}
\begin{document}
\maketitle
\begin{abstract}
\noindent
  Building on the work of Meredith and Radestock
  \cite{DBLP:journals/entcs/MeredithR05} defining a reflective higher
  order mobile process calculus, we present two different systems of
  concurrent combinators into which we can give a full and faithful
  encoding of the rho-calculus. Unlike Yoshida's concurrent
  combinators, which endure a dependency on names, the new operator,
  and replication, both presentations are name-free and can encode a
  $\mathsf{Y}$-combinator-like term that serves in the role of
  replication, and thus have more in common with traditional
  applicative algebra presentations such as the famous $\mathsf{SKI}$
  presentations of the $\lambda$-calculus.

\end{abstract}
\section{Introduction}
  
  The {\pic} (\cite{milner91polyadicpi}) is not a closed theory, but
rather a theory dependent upon some theory of names. Taking an
operational view, one may think of the {\pic} as a procedure that when
handed a theory of names provides a theory of processes that
communicate over those names. This openness of the theory has been
exploited in {\pic} implementations, like the execution engine in
Microsoft's Biztalk \cite{biztalk}, where an ancillary binding
language providing a means of specifying a `theory' of names; e.g.,
names may be tcp/ip ports or urls or object references, etc. But,
foundationally, one might ask if there is a closed theory of
processes, i.e. one in which the theory of names arises from and is
wholly determined by the theory of processes. Behind this question
lurk a whole host of other exciting and potentially enlightening
questions regarding the role of names with structure in calculi of
interaction and the relationship between the structure of names and
the structure of processes.

\subsection{The reflective higher-order $\pi$-calculus}

The $\pi$-calculus was invented in the early 1990s by Robin Milner as
a model of networks of processes with a dynamically changing topology;
two processes initially unaware of each other can be introduced by a
third process.  The reflective higher order $\pi$-calculus uses quoted
processes as names; the term constructors for quote and eval replace
the more traditional nu and replicate constructors.  We also add a
``comm'' term to restrict the contexts in which reduction can occur
\cite{DBLP:journals/corr/StayM15}.

Here is a presentation of the multisorted lambda theory Th(RHOpi) for the reflective higher-order $\pi$-calculus:
\begin{center}
  \begin{longtable}{|p{0.3\linewidth}|p{0.7\linewidth}|}
    \hline
    Sorts:
    \begin{itemize}
      \item $N$ for names
      \item $P$ for processes
    \end{itemize}\bigskip
    Term constructors:
    \begin{itemize}
      \item $\send\maps N \times P^* \to P$
      \item \raggedright $\recv\maps N \times (N^* \Rightarrow P) \to P$
      \item $|\maps P^2 \to P$
      \item $0\maps 1 \to P$
      \item $\comm\maps 1 \to P$
      \item $\quote{-}\maps P \to N$
      \item $\deref{-}\maps N \to P$
    \end{itemize}
    &
    Rewrites:
    \begin{itemize}
      \item $\alpha\maps (p_1 | p_2) | p_3 \Rightarrow p_1 | (p_2 | p_3)$
      \item $\beta\maps p_1 | p_2 \Rightarrow p_2 | p_1$
      \item $\iota\maps 0 | p \Rightarrow p$
      \item \raggedright $\chi\maps \send(x, p_1, \ldots, p_n) \;|\; \recv(x, q) \;|\; \comm \Rightarrow q(\quote{p_1}, \ldots, \quote{p_n}) \;|\; \comm$
      \item $\epsilon\maps \deref{\quote{p}} \Rightarrow p$
    \end{itemize}
    Equations:
    \begin{itemize}
      \item \tshortstack[l]{$\alpha = P^3, \beta = P^2, \iota = P$ ($|$ and 0 form a \\ commutative monoid)}
      \item \tshortstack[l]{$\epsilon = P$ (evaluating a quoted process is the \\ same as the process itself)}
    \end{itemize}\\
    \hline
  \end{longtable}
\end{center}
The simplest RHOpi processes are 0, the ``do nothing'' process; and comm, a ``catalyst'' process that enables communication on a channel.  The only rewrite that is not an identity is $\chi,$ the communication event.  The $\chi$ rewrite is neither confluent nor deterministic.  For example, we can model contention for resources with the term
\[ \recv(x, P)\;|\;\recv(x, Q)\;|\;\send(x,R)\;|\;\comm \]
which has two rewrites out of it, one where the continuation $P$ is invoked on the name $\quote{R}$ and the other where the continuation $Q$ is invoked on it.  We can model message arrival order nondeterminism with the term
\[ \send(x, P)\;|\;\send(x, Q)\;|\;\recv(x,R)\;|\;\comm \]
which has two similar rewrites out of it, one where the continuation $P$ is invoked on the name $\quote{R}$ and one where $P$ is invoked on the name $\quote{Q}$.

In the theory above, comm is preserved by the rewrites; one can think of each comm instance as representing a processor.  An alternative would be to consume comm in the $\chi$ rewrite; then comm would track clock ticks; an application of a consumable comm is the formal verification of billing code for tracking compute resources.

Replication of processes, and therefore general recursion, can be encoded \cite{DBLP:journals/entcs/MeredithR05} via
\[D(x) = \recv(x, y\mapsto \send(x, \deref{y}) | \deref{y})\]
\[!P = \send(x, D(x) | P) | D(x).\]

Caires' \cite{Caires} operator $\triangleright$ for rely-guarantee properties, the adjunct to $|,$ is an instance of our generic modal operator, where $u \triangleright v = u \langle - | - \rangle v:$
\[ \interp{u \triangleright v} = \{ t \;|\; \exists \rho\maps (t\;|\;u) \Rightarrow v\} \]

The logic is generated as above, adding modal operators and recursion to the term language, then adding the Boolean algebra; it is equivalent to the logic we generated by hand in \cite{DBLP:conf/tgc/MeredithR05}.  As an example of a formula in this logic, here is a one-line formula for a process that will always be able to handle another message:
\[ \mu X. \recv(\top, x \mapsto X)\;|\;\top. \]
It says that the process must factor into a piece that is waiting for a message on some channel, and when it receives that message, the continuation will have the same form.

\subsubsection{Namespaces}

When we add a copy of BoolAlg to RHOpi for both $N$ and $P,$ we can write down formulae that talk about sets of names.  The formula $\top$ denotes the set of all names; the formula $\quote{\top}$ denotes the set of names that are quoted processes.  These two sets differ when we have a nonempty set of generating names.

An application of namespace logic is a formula for a ``firewall'': any process satisfying this formula receives no messages except on a name in $\quote{\phi}.$
\[ \mu X. \recv(\quote{\phi}, x \mapsto (X \lor 0)\;|\;\neg\recv(\quote{\neg \phi}, \top))\;|\;\neg\recv(\quote{\neg \phi}, \top) \]
It is, perhaps, surprising that this can be accomplished with a compile-time check, since we usually think of firewalls as a dynamic check.  An application of namespaces to statically proving security properties of code in an object capabilities language can be found in \cite{DBLP:journals/corr/MeredithSD13}.

\section{Conclusion and future work}
We have presented an algorithm for generating a logic from three data:
a 2-Lawvere theory describing a notion of computation, a monad for
describing a notion of collection, and a distributive law. As
mentioned in the introduction, the picture emerging from this view of
logic is that logically useful collections of individuals look like
nothing so much as the individuals that represent the formulae that
collect them.  The range of applications of this algorithm is quite
broad.  Any of the languages given semantics in the K Framework,
including C/C++, Java, Javascript, and many others can automatically
be equipped with logics that serve as a basis for of program
specification and automatically checked verification, and
simultaneously as a basis equipping these languages with new and
powerful type systems.  This considerably lowers the barrier to
embuing popular untyped languages with type systems.  Likewise, the
algorithmic nature of the approach also serves as an aid in future
language design.  For example, Synereo is designing a language for the
smart contracts on the blockchain based on the
$\mathsf{RHO}$-calculus \cite{rholang}.  Equipping this language with a type system
amounts to nothing more than deciding on the type of collection one
wants to collect type inhabitants in.

\bibliographystyle{amsplain}
\bibliography{ladl}
\end{document}
